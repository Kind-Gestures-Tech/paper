\documentclass{article}
\usepackage[utf8]{inputenc}
\usepackage{natbib}
\usepackage{graphicx}
\usepackage{hyperref}

\title{Technical Implementation Draft for Kind Health System}
\author{Saransh Sharma}
\date{Version 1.0 \\ \today}

\begin{document}

\maketitle

\section{Introduction}

This document outlines the technical implementation strategy for the Kind Health system, focusing on leveraging Large Language Models (LLMs) to enhance patient care through a multi-agent, modular architecture. Our approach emphasizes adaptability, modularity, and scalability, ensuring the system remains at the forefront of healthcare technology.

\section{System Architecture}

The system is designed around a core set of principles: modularity, reusability, and loose coupling, enabling flexibility and ease of maintenance. The architecture comprises several key components:

\begin{itemize}
    \item \textbf{Agent Framework}: Utilizes Langroid for creating a robust multi-agent environment.
    \item \textbf{Evaluation Engine}: Incorporates Deep Eval SDK for dynamic, rule-based system evaluation.
    \item \textbf{Tracing Mechanism}: A custom-built tracing system providing transparency and explainability.
    \item \textbf{User Interface}: An intuitive UI integrating system functionalities for end-users, particularly healthcare professionals.
\end{itemize}

\section{Agent Framework}

The agent framework is the backbone of our system, facilitating the interaction between different autonomous agents, each with specific roles and responsibilities.

\subsection{Agent Classes}

\begin{verbatim}
class Agent:
    def __init__(self, name, role):
        self.name = name
        self.role = role
        self.tasks = []

    def receive_task(self, task):
        self.tasks.append(task)

    def execute_tasks(self):
        for task in self.tasks:
            # Task execution logic
            pass
\end{verbatim}

\subsection{Agent Types}

\begin{itemize}
    \item \textbf{Root Agent}: Oversees other agents, ensuring ethical guidelines are followed.
    \item \textbf{Delegation Agent}: Specializes in medical terminology and communication.
    \item \textbf{Doctor Agent}: Stores and manages medical knowledge and feedback.
    \item \textbf{Retrieval Augmented Generation Agent}: Enhances system capabilities by integrating various data sources.
\end{itemize}

\section{Evaluation Engine}

The evaluation engine is critical for maintaining system accuracy, relevance, and efficiency.

\subsection{Evaluation Metrics}

\begin{verbatim}
class EvaluationMetric:
    def __init__(self, name, threshold, calculation_method):
        self.name = name
        self.threshold = threshold
        self.calculation_method = calculation_method

    def evaluate(self, data):
        # Evaluation logic based on calculation_method
        return score
\end{verbatim}

\subsection{Sample Metrics}

\begin{itemize}
    \item Explainability and Trust
    \item Efficiency in Patient Encounters
    \item Concordance with Physician Decisions
\end{itemize}

\section{Tracing Mechanism}

The tracing mechanism provides visibility into the decision-making process, crucial for debugging, optimization, and explainability.

\subsection{Tracing Implementation}

\begin{verbatim}
class Tracer:
    def __init__(self):
        self.traces = []

    def add_trace(self, trace):
        self.traces.append(trace)

    def get_trace(self, identifier):
        # Retrieve specific trace
        return trace
\end{verbatim}

\section{User Interface}

The user interface ties together the system components, offering a seamless experience for healthcare professionals.

\subsection{UI Components}

\begin{verbatim}
class Dashboard:
    def display_patient_data(self, patient_id):
        # Display patient-specific data and insights

class AgentControlPanel:
    def monitor_agent_status(self):
        # Real-time monitoring of agent activities

class EvaluationReport:
    def generate_report(self, metrics):
        # Generate evaluation reports based on selected metrics
\end{verbatim}

\section{Conclusion}

This document provides a foundational blueprint for the technical implementation of the Kind Health system. The modular and scalable architecture ensures the system can evolve with technological advancements and changing healthcare needs.

\end{document}
